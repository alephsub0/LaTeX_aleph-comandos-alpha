\documentclass[11pt,a4paper]{ltxdoc} 
\usepackage[spanish,es-noindentfirst,es-tabla]{babel}

\usepackage[utf8]{inputenc}
\usepackage[T1]{fontenc}
\usepackage{graphicx}
\usepackage{float}
\usepackage[margin=2.5cm,left=3.5cm]{geometry}

\usepackage{mathpazo}
\usepackage{longtable}
\usepackage{aleph-comandos}

\setlength{\parskip}{0.2\baselineskip}
\renewcommand{\baselinestretch}{1.1}

\newcommand{\file}[1]{\texttt{#1}}
\newcommand{\option}[1]{\texttt{#1}}
\newcommand{\package}[1]{\texttt{#1}}
\newcommand{\ejemplo}[2]{\fbox{#1}\hspace{1cm}\fbox{$#2$}}

\title{\file{aleph-comandos.sty}}
\author{Andr\'es Merino}
\date{2019-12-17}

\begin{document}
 
\maketitle
 
\begin{abstract}
    \file{aleph-comandos.sty} es un paquete creado para recopilar varios comandos de uso común entre los colegas de Andrés Merino, dentro de su proyecto personal Alephsub0.
\end{abstract}

\section{Introducción}

El paquete \file{aleph-comandos.sty} es parte del conjunto de clases y paquetes creados por Andrés Merino dentro de su proyecto personal Alephsub0. Este paquete está basado en el paquete \file{comandosEPN.sty} del mismo autor y se cambió su nombre para continuar con el mantenimiento del mismo dentro del proyecto Alephsub0.
    
El paquete provee de una variedad de comandos generados por Juan Carlos Trujillo, Jonathan Ortiz y Andrés Merino, que facilitan la escritura matemática.

\section{Uso}

Para cargar la clase se utiliza: \cs{usepackage}|{aleph-comandos}|.

\section{Comandos}

\subsection{Comandos de función}

\DescribeMacro{\funcion} 
    El comando \cs{funcion} tiene 5 argumentos en el formato\\
        \hspace*{3em}\cs{funcion}\marg{nombre}\marg{dominio}\marg{conjunto de llegada}\marg{variable}\marg{ley de asignación},\\
    con esto, la función genera
    \begin{center}
        |\funcion{f}{A}{B}{x}{f(x)}| 
        \hspace{2cm}
        \fbox{$\funcion{f}{A}{B}{x}{f(x)}$}
    \end{center}
    
\DescribeMacro{\func} 
    El comando \cs{func}  tiene 3 argumentos en el formato\\
        \hspace*{3em}\cs{func}\marg{nombre}\marg{dominio}\marg{conjunto de llegada},\\
    con esto, la función genera
    \begin{center}
        |\func{f}{A}{B}| \hspace{2cm}\fbox{$\func{f}{A}{B}$}
    \end{center}
    

\subsection{Conjuntos}

A continuación se detallan las definiciones de conjuntos:
    \begin{center}
    \begin{tabular}{ccc}
    \hline
    Comando & Resultado & Conjunto\\\hline
        |\N| & $\N$ & Números naturales\\
        |\Nbb| & $\Nbb$ & Números naturales \\
        |\Z| & $\Z$ & Números enteros \\
        |\Zbb| & $\Zbb$ & Números enteros \\
        |\Q| & $\Q$ & Números racionales \\
        |\Qbb| & $\Qbb$ & Números racionales \\
        |\R| & $\R$ & Números reales \\
        |\Rbb| & $\Rbb$ & Números reales \\
        |\reales| & $\reales$ & Números reales \\
        |\C| & $\C$ & Números complejos \\
        |\Cbb| & $\Cbb$ & Números complejos \\
        |\Ibb| & $\Ibb$ & Números irracionales \\
        |\K| & $\K$ & Campo \\
        |\Kbb| & $\Kbb$ & Campo \\
        |\Pbb| & $\Pbb$ & Primos\\
        |\Pol| & $\Pol$ & Polinomios \\
        |\M| & $\M$ & Matrices \\
        \hline
    \end{tabular}
    \end{center}

A pesar de las definiciones para matrices y polinomios, la notación recomendada es:
\begin{itemize}
    \item $\R_n[x]$: para polinomios de grado menor igual que $n$ a coeficientes reales en la variable $x$;
    \item $\R^{n\times m}$: para matrices de orden $n\times m$ a coeficientes reales.
\end{itemize}

\DescribeMacro{\Mat}
    Para este último se define el comando \cs{Mat} con dos argumentos obligatorios y uno opcional, con la siguiente sintaxis:\\
        \hspace*{3em}\cs{Mat}\oarg{coeficiente}\marg{no. filas}\marg{no. columnas},\\
    con esto, el comando genera
    \begin{center}
        |\Mat{3}{1}|
        \hspace{2cm}
        \fbox{$\Mat{3}{1}$}
    \end{center}
    \begin{center}
        |\Mat[\Q]{3}{1}|
        \hspace{2cm}
        \fbox{$\Mat[\Q]{3}{1}$}
    \end{center}
    
\subsection{Operadores}
A continuación se detallan las definiciones de operadores matemáticos:
    \begin{center}
        \begin{longtable}{ccc}
        \hline
        Comando & Resultado & Operador  \\ \hline
        \endfirsthead
        \hline
        Comando & Resultado & Operador  \\ \hline
        \endhead
        \hline 
        \endfoot
        \hline
        \endlastfoot
        |\dom| & $\dom$ & Dominio \\
        |\Dom| & $\Dom$ & Dominio \\
        |\rec| & $\rec$ & Recorrido \\
        |\Rec| & $\Rec$ & Recorrido \\
        |\img| & $\img$ & Imagen \\
        |\Img| & $\Img$ & Imagen \\
        |\rg| & $\rg$ & Rango de una matriz \\
        |\rang| & $\rang$ & Rango de una matriz \\
        |\adj| & $\adj$ & Matriz adjunta \\
        |\cof| & $\cof$ & Matriz de cofactores \\
        |\proy| & $\proy$ & Proyección \\
        |\norm| & $\norm$ & Componente normal \\
        |\inte| & $\inte$ & Interior de un conjunto \\
        |\sin| & $\sin$ & Seno \\
        |\arccsc| & $\arccsc$ & Arcocosecante \\
        |\arccot| & $\arccot$ & Arcocotangente \\
        |\arcsec| & $\arcsec$ & Arcosecante \\
        |\spn| & $\spn$ & Espacio generado \\
        |\gen| & $\gen$ & Espacio generado \\
        |\im| & $\im$ & Parte imaginaria \\
        |\re| & $\re$ & Parte real \\
        |\graf| & $\graf$ & Gráfico de una función \\
        |\sgn| & $\sgn$ & Signo \\
        |\CVA| & $\CVA$ & Conjunto de valores admisibles \\
        |\sol| & $\sol$ & Conjunto solución \\
        |\Sol| & $\Sol$ & Conjunto solución \\
        |\Cis| & $\Cis$ & Operador cis ($\cos + i \sen$) \\
        |\cis| & $\cis$ & Operador cis ($\cos + i \sen$) \\
        |\diam| & $\diam$ & Diámetro \\
        |\Var| & $\Var$ & Varianza \\
        |\Tr| & $\Tr$ & Traza \\
        |\tr| & $\tr$ & Traza \\
        |\mcd| & $\mcd$ & Máximo común divisor \\
        |\mcm| & $\mcm$ & Mínimo común múltiplo \\
        |\dive| & $\dive$ & Divergencia \\
        |\rot| & $\rot$ & Rotacional \\
        |\partes| & $\partes$ & Partes de un conjunto\\
        \hline
        \end{longtable}
    \end{center}


\subsection{Operadores como comandos}
\DescribeMacro{\cl}
    El comando \cs{cl} tiene 1 argumento en el formato\\
        \hspace*{3em}\cs{cl}\marg{conjunto},\\
    con esto, el comando genera
    \begin{center}
        |\cl{A}|
        \hspace{2cm}
        \fbox{$\cl{A}$}
    \end{center}

\DescribeMacro{\norma}
    El comando \cs{norma} tiene 1 argumento en el formato\\
        \hspace*{3em}\cs{norma}\marg{vector},\\
    con esto, el comando genera
    \begin{center}
        |\norma{x}|
        \hspace{2cm}
        \fbox{$\norma{x}$}
    \end{center}
    Si el argumento se lo deja vacío, este genera:
    \begin{center}
        |\norma{}|
        \hspace{2cm}
        \fbox{$\norma{}$}
    \end{center}
    
\DescribeMacro{\prodinner}
    El comando \cs{prodinner} tiene dos argumentos en el formato\\
        \hspace*{3em}\cs{prodinner}\marg{vector 1}\marg{vector 2}, \\
    con esto, el comando genera
    \begin{center}
        |\prodinner{x}{y}|
        \hspace{2cm}
        \fbox{$\prodinner{x}{y}$}
    \end{center}
    Si los argumentos se los deja vacíos, el comando genera:
    \begin{center}
        |\prodinner{}{}|
        \hspace{2cm}
        \fbox{$\prodinner{}{}$}
    \end{center}

\DescribeMacro{\conjugado}
    El comando \cs{conjugado} tiene 1 argumento en el formato\\
        \hspace*{3em}\cs{conjugado}\marg{número},\\
    con esto, el comando genera
    \begin{center}
        |\conjugate{z}|
        \hspace{2cm}
        \fbox{$\conjugate{z}$}
    \end{center}
    
\DescribeMacro{\parcial}
    El comando \cs{parcial} tiene dos argumentos en el formato\\
        \hspace*{3em}\cs{parcial}\marg{función}\marg{variable}, \\
    con esto, el comando genera
    \begin{center}
        |\parcial{f}{x}|
        \hspace{2cm}
        \fbox{$\parcial{f}{x}$}
    \end{center}

\DescribeMacro{\derivada}
    El comando \cs{derivada} tiene dos argumentos en el formato\\
        \hspace*{3em}\cs{derivada}\marg{función}\marg{variable}, \\
    con esto, el comando genera
    \begin{center}
        |\derivada{f}{x}|
        \hspace{2cm}
        \fbox{$\derivada{f}{x}$}
    \end{center}
    
Para más comandos útiles con respecto a derivadas, se puede utilizar el paquete |cool| (\url{https://ctan.org/pkg/cool}).

\subsection{Abreviaciones}
A continuación se detallan las abreviaciones que sirven únicamente en modo matemático.
    \begin{center}
        \begin{tabular}{ccc}
        \hline
            Comando & Resultado & Operador  \\ 
        \hline
            |\setminus| & $\setminus$ & Diferencia de conjuntos pequeña \\
            |\sset| & $\sset$ & Contenencia de conjuntos con igual \\
            |\emptyset| & $\emptyset$ & Conjunto vacío \\
            |\vepsilon| & $\varepsilon$ & Épsilon \\
            |\texty| & $.\texty.$ & Texto ``y'' con espacio \\
            |\yds| & $.\yds.$ & Texto ``y'' con espacio \\
            |\texto| & $.\texto.$ & Texto ``o'' con espacio \\
            |\ods| & $.\ods.$ & Texto ``o'' con espacio \\
            |\siysolosi| & $.\siysolosi.$ & Texto ``si y solo si'' con espacio\\
            |\ssi| & $.\siysolosi.$ & Texto ``si y solo si'' con espacio\\
            |\degre| & $\degre$ & Grados \\
            |\grad| & $\grad$ & Grados \\
            \hline
        \end{tabular}
\end{center}


\subsection{Comandos desplegados}

\DescribeMacro{\dlim}
    El comando \cs{dlim} funciona como una abreviación de \cs{displaystyle}\cs{lim}
    \begin{center}
        |\dlim_{x \to a} f(x)|
        \hspace{2cm}
        \fbox{$\dlim_{x \to a} f(x)$}
    \end{center}

\DescribeMacro{\Lim}
    El comando \cs{Lim} funciona como una abreviación de \cs{displaystyle}\cs{lim}
    \begin{center}
        |\Lim_{x \to a} f(x)|
        \hspace{2cm}
        \fbox{$\Lim_{x \to a} f(x)$}
    \end{center}

\DescribeMacro{\dsum}
    El comando \cs{dsum} funciona como una abreviación de \cs{displaystyle}\cs{sum}
    \begin{center}
        |\dsum_{i=0}^{n} x_i|
        \hspace{2cm}
        \fbox{$\dsum_{i=0}^{n} x_i$}
    \end{center}

\DescribeMacro{\Sum}
    El comando \cs{dsum} funciona como una abreviación de \cs{displaystyle}\cs{sum}
    \begin{center}
        |\Sum_{i=0}^{n} x_i|
        \hspace{2cm}
        \fbox{$\Sum_{i=0}^{n} x_i$}
    \end{center}

\DescribeMacro{\Binom}
    El comando \cs{Binom} funciona como una abreviación de \cs{displaystyle}\cs{binom}
    \begin{center}
        |\Binom{n}{k}| 
        \hspace{2cm}
        \fbox{$\Binom{n}{k}$}
    \end{center}
    
\DescribeMacro{\dint}
    El comando \cs{dint} funciona como una abreviación de \cs{displaystyle}\cs{int}
    \begin{center}
        |\dint_a^b f |
        \hspace{2cm}
        \fbox{$\dint_a^b f $}
    \end{center}

\DescribeMacro{\Int}
    El comando \cs{dint} funciona como una abreviación de \cs{displaystyle}\cs{int}
    \begin{center}
        |\Int_a^b f |
        \hspace{2cm}
        \fbox{$\Int_a^b f $}
    \end{center}


\subsection{Abreviaciones de operadores lógicos}
A continuación se detallan las abreviaciones de operadores lógicos que sirven únicamente en modo matemático.
\begin{center}
    \begin{tabular}{ccc}
    \hline
        Comando & Resultado & Operador \\
    \hline
        |\Di| & $\Di$ & Doble implicación \\
        |\dimp| & $\dimp$ & Doble implicación \\
        |\Dimp| & $\Dimp$ & Doble implicación \\
        |\imp| & $\imp$ & Implicación \\
        |\Imp| & $\Imp$ & Implicación \\
        |\qland| & $.\qland.$ & Conjunción con espacio\\
        |\qlor| & $.\qlor.$ & Disyunción con espacio \\
        |\orm| & $.\orm.$ & \\
        |\andm| & $.\andm.$ & Disyunción con espacio \\
        |\V| & $\V$ & Tautología \\
        |\F| & $\F$ & Contradicción\\
    \hline
    \end{tabular}
\end{center}

\subsection{Delimitadores}

Para delimitadores, se ulitizan las siguientes abreviaciones
\begin{center}
    \begin{tabular}{cc}
    \hline
        Comando & Acción \\
    \hline
        |\r| & |\right| \\
        |\l| & |\left| \\
    \hline
    \end{tabular}
\end{center}

Además, para delimitar intervalos mediante la notación de corchetes abiertos se utilizan las siguientes abreviaciones
\begin{center}
    \begin{tabular}{cc}
    \hline
        Comando & Acción \\
    \hline
        |\rop| & |\right[| \\
        |\lop| & |\left]| \\
        |\rcl| & |\right]| \\
        |\lcl| & |\left[| \\
    \hline
    \end{tabular}
\end{center}

\DescribeMacro{\open}
\DescribeMacro{\openl}
\DescribeMacro{\openr}
\DescribeMacro{\close}
    Finalmente, en intervalos, se utilizan los comandos \cs{open}, \cs{openl}, \cs{openr} y \cs{close}, todos con un argumento obligatorio bajo la misma sintaxis que es\\
        \hspace*{3em}\cs{open}\marg{extremos},\\
    obteniendo
    \begin{center}
        |\open{a,b} |
        \hspace{2cm}
        \fbox{$\open{a,b}$}
    \end{center}
    \begin{center}
        |\openl{a,b} |
        \hspace{2cm}
        \fbox{$\openl{a,b}$}
    \end{center}
    \begin{center}
        |\openr{a,b} |
        \hspace{2cm}
        \fbox{$\openr{a,b}$}
    \end{center}
    \begin{center}
        |\close{a,b} |
        \hspace{2cm}
        \fbox{$\close{a,b}$}
    \end{center}
    
\subsection{Sucesiones}

\DescribeMacro{\suc}
    El comando \cs{suc} tiene un argumento obligatorio (nombre de la sucesión) y uno opcional (índice, por defecto, $n$) en el formato\\
        \hspace*{3em}\cs{suc}\oarg{índice}\marg{término general de la sucesión}, \\
    con esto, el comando genera
    \begin{center}
        |\suc{x_n}|
        \hspace{2cm}
        \fbox{$\suc{x_n}$}
    \end{center}
    o
    \begin{center}
        |\suc[k]{x_k}|
        \hspace{2cm}
        \fbox{$\suc[k]{x_k}$}
    \end{center}
\DescribeMacro{\sucl}
    El comando \cs{sucl} es igual al anterior, pero genera llaves para las sucesiones.
    \begin{center}
        |\sucl{x_n}|
        \hspace{2cm}
        \fbox{$\sucl{x_n}$}
    \end{center}
    o
    \begin{center}
        |\sucl[k]{x_k}|
        \hspace{2cm}
        \fbox{$\sucl[k]{x_k}$}
    \end{center}

\subsection{Comentarios}

\DescribeMacro{comentario}
    El comando \cs{comentario} tiene un argumento en el formato \\
        \hspace*{3em} \cs{comentario}\marg{comentario}, \\
    con esto, el comando genera
    \begin{center}
        |\comentario{Texto comentado}|
        \hspace{2cm}
        \fbox{\comentario{Texto comentado}}
    \end{center}

\subsection{Vectores}
A continuación se detallan los comandos usados para vectores canónicos
\begin{center}
    \begin{tabular}{ccc}
    \hline
    Comando & Resultado \\
    \hline
    |\veci| & $\veci$ \\
    |\vecj| & $\vecj$ \\
    |\veck| & $\veck$ \\
    \hline
    \end{tabular}
\end{center}


\subsection{Problemas}

Cualquier problema, por favor reportarlo a\\
mat.andresmerino@gmail.com.

\newpage
\DocInput{aleph-comandos.dtx}

\end{document}